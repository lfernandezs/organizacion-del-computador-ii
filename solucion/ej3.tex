\begin{enumerate}
\item Dos niveles, kernel (0) y usuarie (1)
\item Se le aplica una función a la dirección lógica, que devuelve la física: \\
  %TODO: Introducción 46
\item \begin{itemize}
  \item[D:] \texttt{Dirty}, marca si se escribió en la página
  \item[A:] \texttt{Accessed}, marca si se escribió/leyó la página
  \item[PCD:] \texttt{Page cache disabled}, determina si la página es cacheada
  \item[PWT:] \texttt{Page write-through}, determina el método de caching (set se escribe al cache y a la memoria, unset se escribe al cache hasta liberarlo)
  \item[U/S:] \texttt{User/System}, determina el nivel de privilegio necesario para acceder a esta pagina
  \item[R/W:] \texttt{Read/Write}, determina si esta página es sólo para leer o si se puede escribir
  \item[P:] \texttt{Present}, indica si la página está cargada en memoria
  \end{itemize}
\item Se necesita 1 tabla del directorio y 5 páginas de esa tabla (1 su pd, 1 su pt, 2 del código, y 1 del stack)
\item
\item El TLB es una especie de caché que guarda las ultimas traducciones de virtual a fisico, para ahorrar tiempo si se accede a una misma pagina muchas veces. Hay que flushearlo porque si cambiamos como traducimos virtuales, va a mantener traducciones viejas.
\end{enumerate}
